\documentclass{article}
\usepackage{amsmath}
\usepackage{booktabs}
\usepackage{array}

% Set extra row height and padding
\setlength{\extrarowheight}{4pt} % Adds extra height between rows
\renewcommand{\arraystretch}{1.5} % Increases the height of each row

\begin{document}

\section*{Cost Analysis}

\begin{table}[h!]
\centering
\begin{tabular}{|p{8cm}|p{4cm}|}
\hline
\textbf{Cost Category} & \textbf{Amount} \\
\hline
\textbf{Initial Investment} & \\
\hline
\hspace{10pt} Software License & \$50{,}000 \\
\hline
\hspace{10pt} Hardware Upgrades & \$10{,}000 \\
\hline
\hspace{10pt} Implementation Costs & \$20{,}000 \\
\hline
\hspace{10pt} Training Costs & \$5{,}000 \\
\hline
\hspace{10pt} Utilities Cost (Yearly) & \$1{,}500 \times 12 = \$18{,}000 \\
\hline
\textbf{Marketing Costs} & \$40{,}000 \\
\hline
\textbf{Other Costs} & \\
\hline
\hspace{10pt} Maintenance and Support (Yearly) & \$10{,}000 \\
\hline
\hspace{10pt} Data Storage (Yearly) & \$2{,}000 \\
\hline
\textbf{Total Development Cost} & \textbf{\$155{,}000} \\
\hline
\end{tabular}
\caption{Cost Analysis}
\end{table}

\section*{Benefit Analysis}

\subsection*{1. Increased Sales}
A 10\% increase in annual revenue:
\[
\text{Present revenue} = \$1{,}000{,}000
\]
\[
\text{Increase} = 0.10 \times 1{,}000{,}000 = 100{,}000
\]

\subsection*{2. Customer Satisfaction}
50\% of new customers become regular customers:
\[
\text{New customers contributing} = 0.50 \times \text{new customers} \times 5000
\]

\subsection*{3. Reduced Labor Costs}
Replacing 3 workers, each paid \$30/hour:
Assuming 40 hours/week, 52 weeks/year:
\[
\text{Savings per worker} = 30 \times 40 \times 52 = 62{,}400
\]
\[
\text{Total Savings} = 62{,}400 \times 3 = 187{,}200
\]

\subsection*{4. Increased Brand Value}
Assuming a 25\% increase in brand value will contribute additional revenue. This is difficult to quantify exactly but may contribute to customer loyalty and new customer acquisition.

\subsection*{5. Adjust for Dollar Rate Decrease}
Each year, the value of the dollar decreases by 15\%. This affects both costs and benefits, but we'll assume it's more relevant to the recurring costs.

\section*{Payback Period Calculation}

\begin{table}[h!]
\centering
\begin{tabular}{|p{6cm}|p{2cm}|p{2cm}|p{2cm}|p{2cm}|p{2cm}|p{2cm}|}
\hline
\textbf{Cash Flow Description} & \textbf{Year 0} & \textbf{Year 1} & \textbf{Year 2} & \textbf{Year 3} & \textbf{Year 4} & \textbf{Year 5} \\
\hline
\textbf{Cost} & \$155{,}000 & \$12{,}000 & \$12{,}000 & \$12{,}000 & \$12{,}000 & \$12{,}000 \\
\hline
\textbf{Benefit} & \$0 & \$780{,}000 & \$780{,}000 & \$780{,}000 & \$780{,}000 & \$780{,}000 \\
\hline
\textbf{Net Cash Flow} & (\$155{,}000) & \$768{,}000 & \$768{,}000 & \$768{,}000 & \$768{,}000 & \$768{,}000 \\
\hline
\textbf{Cumulative Cash Flow} & (\$155{,}000) & \$613{,}000 & \$1{,}381{,}000 & \$2{,}149{,}000 & \$2{,}917{,}000 & \$3{,}685{,}000 \\
\hline
\end{tabular}
\caption{Payback Period Calculation}
\end{table}

\section*{Payback Period Determination}

The cumulative cash flow becomes positive after the first year, therefore the payback period is:

\textbf{Payback Period: 1 Year}

\section*{ROI Analysis}

The ROI (Return on Investment) can be calculated using the following formula:

\[
\text{ROI} = \frac{\text{Total Benefits} - \text{Total Costs}}{\text{Total Costs}} \times 100
\]

\begin{align*}
\text{Total Costs} &= \$155{,}000 \text{ (Initial)} + \$12{,}000 \text{ (Ongoing Year 1)} \\
&= \$167{,}000 \\
\text{Total Benefits (Year 1)} &= \$780{,}000 \\
\text{ROI} &= \frac{780{,}000 - 167{,}000}{167{,}000} \times 100 \approx 466.47\%
\end{align*}

\textbf{ROI for Year 1: 466.47\% and Payback Period: 1 Year}
\newline

This analysis shows that the investment in the software system will be fully recovered within 1 year, with a high ROI of 466.47\% by the end of the first year.

\end{document}
